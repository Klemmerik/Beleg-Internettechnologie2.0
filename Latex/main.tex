% main.tex - Hauptdatei der Vorlage
% Vorlage
%! LaTeX Vorlage
\documentclass[12pt, fleqn, captions=nooneline, titlepage, footsepline, headsepline, toc=chapterentrywithdots, listof=entryprefix, bibliography=totoc, parskip=half-]{scrreprt}

\usepackage{silence} %unnötige Warnungen unterdrücken
\WarningFilter{latex}{You have requested}
\WarningFilter{microtype}{Unable to apply patch}
% \WarningFilter{scrlayer-scrpage}{\headheight to low}
% \WarningFilter{scrlayer-scrpage}{\footheight to low}
% \WarningFilter{scrlayer-scrpage}{Very small head height detected}
% \WarningFilter{fvextra}{} % was caused by loading csquotes before minted (which loads fvextra)
% \WarningFilter{lineno}{}

\pdfsuppresswarningpagegroup=1

\ProvidesPackage{metadaten}
\usepackage{metadaten}

\usepackage{tocbasic}
\usepackage[ngerman]{babel}
\usepackage[%
  backend=biber,
  labeldateparts=true,
  style=authortitle,
  isbn=false,
  dashed=false,
  maxnames=3]{biblatex}

%!  Änderungen für scrreprt
\renewcommand{\autodot}{}
\usepackage{chngcntr}
\counterwithout{figure}{chapter}
\counterwithout{table}{chapter}
\counterwithout{footnote}{chapter}
\RedeclareSectionCommand[style=section,afterskip=.15em]{chapter}
\setcounter{secnumdepth}{\subsubsectionnumdepth}
\setcounter{tocdepth}{\subsubsectionnumdepth}
\addtokomafont{chapter}{\LARGE}
\addtokomafont{section}{\Large}
\addtokomafont{subsection}{\large}
\addtokomafont{subsubsection}{\normalsize}
\renewcommand*{\chaptermarkformat}{}

%! Formatierung aller Verzeichnisse
%! Das Inhaltsverzeichnis wird an dieser Stelle formatiert.
\RedeclareSectionCommands[beforeskip=-.1\baselineskip, afterskip=.1\baselineskip, tocindent=0pt, tocnumwidth=45pt]{chapter, section,subsection,subsubsection}

\renewcaptionname{ngerman}{\refname}{Quellenverzeichnis}
\setuptoc{toc}{totoc}
\setuptoc{lof}{totoc}
\setuptoc{lot}{totoc}
\renewcommand*\listoflofentryname{\bfseries\figurename}
\BeforeStartingTOC[lof]{\renewcommand*\autodot{\space\space\space\space}}
\addtokomafont{captionlabel}{\bfseries}
\renewcommand*\listoflotentryname{\bfseries\tablename}
\BeforeStartingTOC[lot]{\renewcommand*\autodot{\space\space\space\space}}

%! Anhangsverzeichnis
\providecaptionname{ngerman}{\listofatocentryname}{Anhang}

\makeatletter
\AfterTOCHead[atoc]{\let\if@dynlist\if@tocleft}
\newcommand*{\useappendixtocs}{
  \renewcommand*{\ext@toc}{atoc}
  \RedeclareSectionCommands[tocindent=0pt]{chapter, section, subsection, subsubsection}
  \RedeclareSectionCommands[tocnumwidth=85pt]{chapter, section, subsection, subsubsection}
  \renewcommand*\listoflofentryname{\mdseries}
  \renewcommand{\thechapter}{\arabic{chapter}}
  \renewcommand{\@dotsep}{10000}
  }
\newcommand*{\usestandardtocs}{
  \renewcommand*{\ext@toc}{toc}
  }
\makeatother

%! Formelverzeichnis
\DeclareNewTOC[
  type=formel,                         % Name der Umgebung
  types=formeln,                       % Erweiterung (\listofschemes)
  float,                               % soll gleiten
  tocentryentrynumberformat=\bfseries, % voreingestellte Gleitparameter
  name=Formel,                         % Name in Überschriften
  listname={Formelverzeichnis},        % Listenname
  % counterwithin=chapter
]{lom}
\setuptoc{lom}{totoc}
\renewcommand*{\formelformat}{\hfill%
  \formelname~\theformel%
  \autodot{\space\space\space}
}
\BeforeStartingTOC[lom]{\renewcommand*\autodot{\space\space\space\space}}

%! Inhalts-/Anhangsverzeichnis
\DeclareNewTOC[%
  %owner=\jobname,
  tocentrystyle=tocline,
  tocentryentrynumberformat=\PrefixBy{Anhang},
  listname={Anhangsverzeichnis}% Titel des Verzeichnisses
]{atoc}% Dateierweiterung (a=appendix, toc=table of contents)

\usepackage{xpatch}
\xapptocmd\appendix{%
  \useappendixtocs
  \listofatocs
  \addcontentsline{toc}{chapter}{Anhangsverzeichnis}
  \newpage
  \pagenumbering{gobble}
  \pagestyle{scrheadings}
  \clearmainofpairofpagestyles
  \clearplainofpairofpagestyles
  \rohead{\textnormal{Anhang~\arabic{chapter}}}
  \lohead{\textnormal{\currentname}}
  \lofoot{}
  \cofoot{}
  \rofoot{}
}{}{}


%! Ermöglicht die Ausgabe des aktuellen Titels.
\usepackage{nameref}
\makeatletter
\newcommand*{\currentname}{\@currentlabelname}
\makeatother

%! zusätzliche LaTeX-Packages
\usepackage[table]{xcolor}
\usepackage{amsmath}
\usepackage{amssymb}
\usepackage{amsthm}
\usepackage{tabularx}
\usepackage{multirow}
\usepackage{booktabs}
\usepackage{svg}
\usepackage{graphicx}
\usepackage{float}
%! floatplacement für Codes muss in minted.tex definiert werden
\floatplacement{figure}{htbp}
\floatplacement{table}{htbp}
\floatplacement{figure}{htbp}
\floatplacement{formel}{htbp}
\usepackage[a4paper,lmargin={2.5cm},rmargin={2.5cm},tmargin={2cm},bmargin={2cm}]{geometry}
\usepackage{lineno}
\usepackage[T1]{fontenc}
\usepackage{listings}
\usepackage{tikz}
\usepackage{varwidth}
\usepackage{ifthen}
\usepackage{etoolbox}

%! Abbildungen von Verzeichnisstrukturen

\usepackage{dirtree}

%! Centered Dirtree - https://tex.stackexchange.com/posts/100182/revisions
\makeatletter
\def\dirtree#1{%
  %%2012\let
  \DT@indent=\parindent
  \parindent=\z@
  %%2012\let
  \DT@parskip=\parskip
  \parskip=\z@
  %%2012\let
  \DT@baselineskip=\baselineskip
  \baselineskip=\DTbaselineskip
  \let\DT@strut=\strut
  \def\strut{\vrule width\z@ height0.7\baselineskip depth0.3\baselineskip}%
  \DT@counti=\z@
  \let\next\DT@readarg
  \next#1\@nil
  \dimen\z@=\hsize
  \advance\dimen\z@ -\DT@offset
  \advance\dimen\z@ -\DT@width
%  \setbox\z@=\hbox to\dimen\z@{%
  \setbox\z@=\hbox{%
%    \hsize=\dimen\z@
    \vbox{\hbox{\@nameuse{DT@body@1}}}%
  }%
  \dimen\z@=\ht\z@
  \advance\dimen0 by\dp\z@
  \advance\dimen0 by-0.7\baselineskip
  \ht\z@=0.7\baselineskip
  \dp\z@=\dimen\z@
  \par\leavevmode
  \kern\DT@offset
  \kern\DT@width
  \box\z@
  \endgraf
  \DT@countii=\@ne
  \DT@countiii=\z@
  \dimen3=\dimen\z@
  \@namedef{DT@lastlevel@1}{-0.7\baselineskip}%
  \loop
  \ifnum\DT@countii<\DT@counti
    \advance\DT@countii \@ne
    \advance\DT@countiii \@ne
    \dimen\z@=\@nameuse{DT@level@\the\DT@countii}\DT@all
    \advance\dimen\z@ by\DT@offset
    \advance\dimen\z@ by-\DT@all
    \leavevmode
    \kern\dimen\z@
    \DT@countiv=\DT@countii
    \count@=\z@
    %%2012\LOOP
    \DT@loop
      \advance\DT@countiv \m@ne
      \ifnum\@nameuse{DT@level@\the\DT@countiv} >
        \@nameuse{DT@level@\the\DT@countii}\relax
      \else
        \count@=\@ne
      \fi
    \ifnum\count@=\z@
    %%2012\REPEAT
    \DT@repeat
    \edef\DT@hsize{\the\hsize}%
    \count@=\@nameuse{DT@level@\the\DT@countii}\relax
    \dimen\z@=\count@\DT@all
    \advance\hsize by-\dimen\z@
    \setbox\z@=\vbox{\hbox{\@nameuse{DT@body@\the\DT@countii}}}%
    \hsize=\DT@hsize
    \dimen\z@=\ht\z@
    \advance\dimen\z@ by\dp\z@
    \advance\dimen\z@ by-0.7\baselineskip
    \ht\z@=0.7\baselineskip
    \dp\z@=\dimen\z@
    \@nameedef{DT@lastlevel@\the\DT@countii}{\the\dimen3}%
    \advance\dimen3 by\dimen\z@
    \advance\dimen3 by0.7\baselineskip
    \dimen\z@=\@nameuse{DT@lastlevel@\the\DT@countii}\relax
    \advance\dimen\z@ by-\@nameuse{DT@lastlevel@\the\DT@countiv}\relax
    \advance\dimen\z@ by0.3\baselineskip
    \ifnum\@nameuse{DT@level@\the\DT@countiv} <
        \@nameuse{DT@level@\the\DT@countii}\relax
      \advance\dimen\z@ by-0.5\baselineskip
    \fi
    \kern-0.5\DT@rulewidth
    \hbox{\vbox to\z@{\vss\hrule width\DT@rulewidth height\dimen\z@}}%
    \kern-0.5\DT@rulewidth
    \kern-0.5\DT@dotwidth
    \vrule width\DT@dotwidth height0.5\DT@dotwidth depth0.5\DT@dotwidth
    \kern-0.5\DT@dotwidth
    \vrule width\DT@width height0.5\DT@rulewidth depth0.5\DT@rulewidth
    \kern\DT@sep
    \hbox{\box\z@}%
    \endgraf
  \repeat
  \parindent=\DT@indent
  \parskip=\DT@parskip
  %%2012\DT@baselineskip=\baselineskip
  \baselineskip=\DT@baselineskip
  \let\strut\DT@strut
}

\makeatother


%! PageBreaks nach jeder Section
\let\oldchapter\chapter
\renewcommand\chapter{\clearpage\oldchapter}

%! Code-Umgebungen
%! Code-Integration im Dokument
%? Inklusive Erzeugung eines Custom-Enviroments für Programmcodes
\usepackage{minted}

% ! Anpassung der Minted-Umgebung, um Abstände zu vereinheitlichen und die Optik zu verbessern
\let\oldminted\minted
\let\oldendminted\endminted
\def\minted{\begingroup \vspace{-0.3cm} \oldminted}
\def\endminted{\oldendminted \vspace{-0.5cm} \endgroup}

\xapptocmd{\inputminted}{\vspace{-0.5cm}}{}{}

%Zeilennummern neu definieren, um Warnungen zu vermeiden und modern anmutende Zahlen zu nutzen
\renewcommand{\theFancyVerbLine}{\scriptsize{\arabic{FancyVerbLine}}}

%Standardformatierung für minted-Umgebung erstellen
\setminted{
    tabsize=2,
    breaklines,
    autogobble,
    fontfamily=courier,
    linenos,
    %! see https://pygments.org/demo/#try, other nice options: paraiso-light, solarized-light, rainbow_dash, gruvbox-light, stata, tango
    style=emacs,
    fontsize=\footnotesize
}
%Keine Zeilennnummer, wenn der einzeilige \mint-Befehl genutzt wird
\xpretocmd{\mint}{\setminted{linenos=false}}{}{}
\xpretocmd{\minted}{\setminted{linenos=true}}{}{}

\DeclareNewTOC[
  type=code,                           % Name der Umgebung
  types=codes,                         % Erweiterung (\listofschemes)
  float,                               % soll gleiten
  floatpos=htbp,
  tocentryentrynumberformat=\bfseries, % voreingestellte Gleitparameter
  name=Code,                           % Name in Überschriften
  listname={Programmcodeverzeichnis},  % Listenname
  % counterwithin=chapter
]{loc}
\setuptoc{loc}{totoc}

\renewcommand*{\codeformat}{%
  \codename~\thecode%
  \autodot{\space\space\space}
}
\BeforeStartingTOC[loc]{\renewcommand*\autodot{\space\space\space\space}}
\AtBeginEnvironment{minted}{\vspace{\baselineskip}}


\usepackage{scrhack} %necessary to use float with scrreprt
\usepackage{csquotes}

%! Schriftart
%! Die HAWA schreibt Arial vor, welches in Standard LaTeX-Distributionen nicht mitgeliefert wird. Helvetica ist nahezu identisch.
\usepackage{helvet}
\usepackage{microtype}
\renewcommand{\familydefault}{\sfdefault}
%! Hyperref und PDF-meta
\usepackage[hidelinks]{hyperref}

%! PDF-Metadaten (Autor/Titel/Beschreibung)
%! PDF-Metadaten
\hypersetup{pdftitle={\titel}}
\hypersetup{pdfsubject={\kurzbeschreibung}}
\ifthenelse{\isundefined{\autorzwei}}{\hypersetup{pdfauthor={\autoreins}}}{%
    \ifthenelse{\isundefined{\autordrei}}{\hypersetup{pdfauthor={\autoreins, \autorzwei}}}{%
        \ifthenelse{\isundefined{\autorvier}}{\hypersetup{pdfauthor={\autoreins, \autorzwei, \autordrei}}}{%
        \hypersetup{pdfauthor={\autoreins, \autorzwei, \autordrei, \autorvier}}
        }
    }
}

\usepackage[numbered]{bookmark}
\usepackage[printonlyused]{acronym}
\usepackage{enumitem}

%Abkürzungsverzeichnis - Formatierung (bspw. zur korrekten Anzeige von BASH-Befehlen)
\renewcommand*{\aclabelfont}[1]{\acsfont{#1}}

%! Untertitel/Legenden
\usepackage{caption}
% Captions linksbündig, auch wenn einzeilig
\captionsetup{
  labelsep=none,
  justification=raggedright,
  singlelinecheck=false
}
\renewcommand*{\figureformat}{%
  \figurename~\thefigure%
  \autodot{\space\space\space}
}
\renewcommand*{\tableformat}{%
  \tablename~\thetable%
  \autodot{\space\space\space}
}


%! Literaturverzeichnis und Zitierbefehle
%! Formatierung des Literaturverzeichnis
\DeclareLabeldate{%
\field{year}
}

\DeclareExtradate{%
\scope{
\field{labelyear}
\field{year}}
}
\DeclareFieldFormat{url}{In: \url{#1}}
\DeclareFieldFormat{urldate}{\space\mkbibparens{#1}}
\DeclareFieldFormat{urlday}{\forcezerosmdt{#1}}
\DeclareFieldFormat{urlmonth}{\forcezerosmdt{#1}}
\DeclareFieldFormat{urlyear}{\forcezerosmdt{#1}}
\DeclareFieldFormat{date}{#1\printfield{extradate}}

\urlstyle{same}
%? Author-Format für Literaturverzeichnis
\DeclareNameFormat{author}{%
  \nameparts{#1}%
  \usebibmacro{name:family-given}
    {\expandafter\ifblank\expandafter{\namepartgiven}
       {\namepartfamily}% no family name, don't uppercase
       {\MakeUppercase{\namepartfamily}}%
    }
    {\namepartgiven}
    {\namepartprefix}
    {\namepartsuffix}%
    \ifthenelse{\value{listcount}=1\AND\ifmorenames}{\andothersdelim\bibstring{andothers}}{}%
}
%? Author-Format für Fußnoten
\DeclareNameFormat{author_fn}{%
\nameparts{#1}%
\usebibmacro{name:family}
  {\expandafter\ifblank\expandafter{\namepartgiven}
     {\namepartfamily}% no family name, don't uppercase
     {\MakeUppercase{\namepartfamily}}%
  }
  {\namepartgiven}
  {\namepartprefix}
  {\namepartsuffix}%
  \ifthenelse{\value{listcount}=1\AND\ifmorenames}{\andothersdelim\bibstring{andothers}}{}%
}
\DeclareFieldFormat{title}{#1}

\renewcommand{\multinamedelim}{\addsemicolon\space}
\renewcommand{\finalnamedelim}{\addsemicolon\space}
\renewcommand{\labelnamepunct}{\addcolon\space}
\renewcommand*{\finentrypunct}{}
\renewcommand{\andothersdelim}{\addsemicolon\space}

\setlength\bibitemsep{\baselineskip}
\setlength\bibhang{0pt}

%! Formatierung der Fußnotenzitate / Literaturverzeichnis
\renewcommand*{\newunitpunct}{\addcomma\space}

%? Normales Zitat...
\DeclareCiteCommand{\zitat}[\mkbibfootnote]
  {\usebibmacro{prenote}}
  {\usebibmacro{citeindex}
   \setunit{\addnbspace}
   \bibhyperref{\printnames[author_fn]{labelname}}
   \setunit{\labelnamepunct}
   \newunit
   \printfield{location}
   \newunit
   \printfield{labelyear}\printfield{extradate}
   \newunit
   \printfield{pages}
   }
  {\addsemicolon\space}
  {\usebibmacro{postnote}}

%? Online Zitat
\DeclareCiteCommand{\onlinezitat}[\mkbibfootnote]
  {\usebibmacro{prenote}}
  {\usebibmacro{citeindex}
   \setunit{\addnbspace}
   online:
   \bibhyperref{\printnames[author_fn]{labelname}}
   \setunit{\labelnamepunct}
   \newunit
   \printfield{labelyear}\printfield{extradate}
   \printtext{(}\printfield{urlday}\printtext{.}\printfield{urlmonth}\printtext{.}\printfield{urlyear}\printtext{)}}
  {\addsemicolon\space}
  {\usebibmacro{postnote}}

%? Sinngemäßes Zitat
\DeclareCiteCommand{\vgzitat}[\mkbibfootnote]
  {\usebibmacro{prenote}}
  {\usebibmacro{citeindex}
   \setunit{\addnbspace}
   vgl. 
   \bibhyperref{\printnames[author_fn]{labelname}}
   \setunit{\labelnamepunct}
   \newunit
   \printfield{location}
   \newunit
   \printfield{labelyear}\printfield{extradate}
   \newunit
   \printfield{pages}
   }
  {\addsemicolon\space}
  {\usebibmacro{postnote}}


%? Sinngemäßes Online Zitat
\DeclareCiteCommand{\vgonlinezitat}[\mkbibfootnote]
  {\usebibmacro{prenote}}
  {\usebibmacro{citeindex}
   \setunit{\addnbspace}
   vgl. online:
   \bibhyperref{\printnames[author_fn]{labelname}}
   \setunit{\labelnamepunct}
   \newunit
   \printfield{labelyear}\printfield{extradate}
   \printtext{(}\printfield{urlday}\printtext{.}\printfield{urlmonth}\printtext{.}\printfield{urlyear}\printtext{)}}
  {\addsemicolon\space}
  {\usebibmacro{postnote}}

%? Unveröffentlichtes Zitat
\DeclareCiteCommand{\uvzitat}[\mkbibfootnote]
  {\usebibmacro{prenote}}
  {\usebibmacro{citeindex}
   \setunit{\addnbspace}
   unveröffentlicht:
   \bibhyperref{\printnames[author_fn]{labelname}}
   \setunit{\labelnamepunct}
   \newunit
   \printfield{location}
   \newunit
   \printfield{labelyear}\printfield{extradate}
   \newunit
   \printfield{pages}
   }
  {\addsemicolon\space}
  {\usebibmacro{postnote}}

%? Sinngemäßes, unveröffentlichtes Zitat
\DeclareCiteCommand{\vguvzitat}[\mkbibfootnote]
{\usebibmacro{prenote}}
{\usebibmacro{citeindex}
 \setunit{\addnbspace}
 vgl. unveröffentlicht:
 \bibhyperref{\printnames[author_fn]{labelname}}
 \setunit{\labelnamepunct}
 \newunit
 \printfield{location}
 \newunit
 \printfield{labelyear}\printfield{extradate}
 \newunit
 \printfield{pages}
 }
{\addsemicolon\space}
{\usebibmacro{postnote}}



\renewcommand{\bibfootnotewrapper}[1]{\bibsentence#1}

\DeclareMultiCiteCommand{\zitate}[\mkbibfootnote]{\footpartcite}{\addsemicolon\space}
\addbibresource{literatur.bib}



%! Kopf- und Fußzeilen
\input{vorlage/vorlage_subs/kopf_fußzeile}

%! verbesserte Umbrüche (hoffentlich)
\input{vorlage/vorlage_subs/umbrüche}

%? wird am Ende geladen, um einheitliche Spacings sicherzustellen
\usepackage{setspace}
%! setzt den Zeilenabstand in Floats (Tabellen, Code-Listings usw.) auf den in LaTeX definierten Wert (statt 1.0)
\makeatletter
\let\@xfloat=\latex@xfloat
\makeatother

% vordefinierte Kommandos der Vorlage
%! Eigene Befehle zur erleichterten Nutzung
% Hilfsbefehle
\newcommand{\fontheightsvg}[1]{\includesvg[height=1.75ex, inkscapelatex=false]{#1}}
\newcommand{\dtfolder}{\fontheightsvg{vorlage/bilder/dirtree_folder}\hspace{0.1cm}}
\newcommand{\dtfile}{\fontheightsvg{vorlage/bilder/dirtree_file}\hspace{0.1cm}}
\newcommand{\dtusb}{\raisebox{.1em}{\fontheightsvg{vorlage/bilder/usb}\hspace{0.1cm}}}

% Umgebungen u.Ä.
\newcommand{\fn}[1]{\footnote{\hspace{0.5em}#1}}
%! #1 - Breite, #2 - Dateiname, #3 Caption, #4 - Label
\newcommand{\bild}[4][1.0]{\begin{figure}
  \centering
  \includegraphics[width=#1\columnwidth]{bilder/#2}
  \caption{#3}
  \label{#4}
  \end{figure}}
\newcommand{\striche}[1]{\glqq #1\grqq{}}
%! #1 - Breite, #2 - Dateiname, #3 Caption, #4 - Label
\newcommand{\svg}[4][1.0]{\begin{figure}
    \centering
    \includesvg[width=#1\columnwidth,inkscapelatex=false]{bilder/#2}
    \caption{#3}
    \label{#4}
    \end{figure}}
%! #1 - Dirtree, #2 - Caption, #3 - Label
\newcommand{\verzeichnis}[3]{\begin{figure}
  % https://tex.stackexchange.com/a/99591/220899
  \renewcommand{\DTstyle}{\textrm\expandafter\raisebox{-0.7ex}}
  \centering
  \begin{varwidth}{\textwidth}
    \dirtree{#1}  
  \end{varwidth}
  \caption{#2}
  \label{#3}
  \end{figure}}
\newcommand{\logisch}[1]{$``#1``$}
\newcommand{\vglink}[2]{\fn{vgl.~\href{#1}{#1}~(#2)}}
% Muss statt \caption in die Umgebung wenn eine Fußnote verwendet werden soll, in Verbindung mit der Zeile darunter
\newcommand{\linkcaption}[1]{\caption[#1]{#1\footnotemark}}
% Muss unter die Umgebung, wenn eine Fußnote in der Umgebung verwendet werden soll
\newcommand{\vgcaption}[2]{\footnotetext{\hspace{0.5em}vgl.~\href{#1}{#1}~(#2)}}
\newcommand{\python}[1]{\mintinline{python}{#1}}
%! #1 - Formel, #2 - Legende, #3 - Caption, #4 - Label
\newcommand{\formula}[4]{\begin{formel}
  \pretocmd{\captionbelow}{\onelinecaptionstrue}{}{}
  \KOMAoptions{captions=centeredbeside}
  \begin{captionbeside}[#3]{\textbf{#3}}[r]#1\end{captionbeside}
  \pretocmd{\captionbelow}{}{}{}#2\label{#4}\end{formel}}

%! Referenzierung
\newcommand{\literef}[1]{\emph{\hyperref[{#1}]{\autoref{#1} - \nameref{#1}}}}
\newcommand{\fullref}[1]{(\emph{\hyperref[{#1}]{siehe \autoref{#1} - \nameref{#1}}})}

% Kompatibilität zu alter "Architektur"
\let\aref\literef
\let\bref\literef
\let\cref\literef
\let\fref\literef
\let\sref\literef
\let\tref\literef
\let\litearef\literef
\let\litebref\literef
\let\litecref\literef
\let\litefref\literef
\let\litesref\literef
\let\litetref\literef
\let\fullaref\fullref
\let\fullbref\fullref
\let\fullcref\fullref
\let\fullfref\fullref
\let\fullsref\fullref
\let\fulltref\fullref

% automatische Ermittlung des ref-Typs nach https://tex.stackexchange.com/questions/33776/get-label-target-type
\makeatletter

% Helper macro to extract the type (section,subsection...) or the type name
% out of the label reference. Works with hyperref only.
% Argument #1 is a macro of form \def\...#1...\@nil{...}
% Argument #2 is the label reference, e.g. "sect:test"
\newcommand*\@autoref[2]{% \HyPsd@@@autoref from hyperref, modified
  \expandafter\ifx\csname r@#2\endcsname\relax
    ??%
  \else
    \expandafter\expandafter\expandafter\@@autoref
        \csname r@#2\endcsname{}{}{}{}\@nil#1\@nil
  \fi
}
\def\@@autoref#1#2#3#4#5\@nil#6\@nil{% \HyPsd@autorefname, modified
  #6#4.\@nil}% Argument #4 = type and number, e.g. "section.1" or "subsection.1.2"

% \reftype results in the type name, e.g. "section" or "figure".
% The starred variant will remove a star, if existent, i.e. "section*" will become "section"
\newcommand\reftype{%
  \@ifstar
    {\@autoref\@@reftype}%
    {\@autoref\@reftype}}
\def\@reftype#1.#2\@nil{#1}
\def\@@reftype#1.#2\@nil{\@@@reftype#1*\@nil}
\def\@@@reftype#1*#2\@nil{#1}

% \autoreftype results in the type prose name (plus space character),
% e.g. "section" in English or "Abschnitt" in German
% (like \autoref, but without number).
% \HyPsd@@autorefname is defined in the hyperref package.
\newcommand*\autoreftype[1]{\@autoref\HyPsd@@autorefname{#1}}

% An alternative version of \autoreftype without space at the end.
% Since the \space is hard coded inside \HyPsd@@autorefname we use our
% own version called \@autoreftype instead.
% Furthermore we offer a starred variant which will work with labels to
% \section* etc., too.
\renewcommand*\autoreftype{%
  \@ifstar
    {\@autoref\@@autoreftype}%
    {\@autoref\@autoreftype}}
\def\@autoreftype#1.#2\@nil{% = \HyPsd@@autorefname without \space
  \ltx@IfUndefined{#1autorefname}%
    {\ltx@IfUndefined{#1name}%
      {}%
      {\csname#1name\endcsname}}%
    {\csname#1autorefname\endcsname}}
\def\@@autoreftype#1.#2\@nil{%
  \expandafter\@autoreftype\@@@reftype#1*\@nil.\@nil}

\makeatother

% Bezeichnungen Unterabschnitt und Unterunterabschnitt durch Abschnitt ersetzen, auskommentieren, falls diese Bezeichner erwünscht sind
\AtBeginDocument{%
  \renewcommand*{\subsectionautorefname}{\sectionautorefname}%
  \renewcommand*{\subsubsectionautorefname}{\sectionautorefname}%
}

% Auf gleiche Art können auch eigene Kommandos in eine Datei wie 'inhalt/kommandos.tex' ausgelagert werden

%Dokument-Anfang
\begin{document}

%! #########################################
%! Inhalt der Arbeit
\frontmatter

%! automatische Auswahl der Titelseite
\ifthenelse{\isundefined{\autorzwei}}{%Titelseite
\begin{titlepage}
\begin{center}
\textbf{\Huge \begin{center}\ifthenelse{\isundefined{\jahr}}{Praxisbeleg}{Bachelorthesis}\end{center}}
\LARGE{\titel \\}
\vspace{1.0cm}
\end{center}
\begin{flushleft}
\large{
\begin{tabular}{l l r}
\vspace{0.7cm}
\textbf{Vorgelegt am:}\quad\quad\quad & \abgabedatum\\
\textbf{Von:}           ~ & \textbf{\autoreins}\\
~ & \autorstrasse \\
\vspace{0.7cm}
~ & \autorort \\
\textbf{Studiengang:}   ~ & \studiengang \\
\vspace{0.7cm}
\textbf{Studienrichtung:} ~ & \studienrichtung \\
\vspace{0.7cm}
\textbf{Seminargruppe:} ~ & \seminargruppe \\
\vspace{0.7cm}
\textbf{Matrikelnummer:} ~ & \matnumeins \\
\textbf{Praxispartner:} ~ & \institutioneins \\
~ & \partnerstrasse \\
\vspace{0.7cm}
~ & \partnerort \\
\textbf{Gutachter:}     ~ & \betreuereins \\ ~ & (\institutioneins)\\
                        ~ & \betreuerzwei \\ ~ & (\institutionzwei)\\
\vspace{0.7cm}
\end{tabular}}
\end{flushleft}
\end{titlepage}
\newpage
}{%Titelseite

\begin{titlepage}
\begin{center}

\textbf{\Huge Projektarbeit}\\
\vspace{1.5cm}
\LARGE{\titel \\}
\vspace{1.5cm}
\end{center}
\begin{flushleft}
\large{
\begin{tabular}{l l r}
\vspace{1.0cm}
\textbf{Vorgelegt am:}\quad\quad\quad & \abgabedatum\\

\textbf{Von:}           ~ & \textbf{\autoreins}\\
\ifthenelse{\isundefined{\autordrei}}{\vspace{1.0cm} ~ & \textbf{\autorzwei}\\}{%
        \ifthenelse{\isundefined{\autorvier}}{~ & \textbf{\autorzwei}\\
        \vspace{1.0cm}
        ~ & \textbf{\autordrei}\\}{~ & \textbf{\autorzwei}\\ ~ & \textbf{\autordrei}\\ \vspace{1.0cm} ~ & \textbf{\autorvier}\\}
    }



\textbf{Studiengang:}   ~ & \studiengang \\
\vspace{1.0cm}
\textbf{Studienrichtung:} ~ & \studienrichtung \\
\vspace{1.0cm}
\textbf{Seminargruppe:} ~ & \seminargruppe \\

\textbf{Matrikelnummer:} ~ & \matnumeins \\
\ifthenelse{\isundefined{\autordrei}}{\vspace{1.0cm}
~ & \matnumzwei \\}{%
        \ifthenelse{\isundefined{\autorvier}}{~ & \matnumzwei \\
        \vspace{1.0cm} ~ & \matnumdrei \\}{~ & \matnumzwei \\ ~ & \matnumdrei \\ \vspace{1.0cm} ~ & \matnumvier \\}
    }

\textbf{Gutachter:}     ~ & \betreuereins \\ ~ & (\institutioneins)\\
                        ~ & \betreuerzwei \\ ~ & (\institutionzwei)\\

\end{tabular}}
\end{flushleft}
\end{titlepage}
\newpage}
\vfuzz=10pt
\ifthenelse{\isundefined{\sperre}}{}{\chapter*{Sperrvermerk}
\rohead{\textnormal{Sperrvermerk}}
\addcontentsline{toc}{chapter}{Sperrvermerk}
\vspace*{0.7cm}

\ifthenelse{\isundefined{\autorzwei}}{\ifthenelse{\isundefined{\jahr}}{Der vorliegende Praxisbeleg}{Die vorliegende Bachelorthesis}}{Die vorliegende Projektarbeit} mit dem Titel:
\vspace*{0.7cm}

\titel


\vspace*{0.7cm}
beinhaltet interne und vertrauliche Informationen des Unternehmens:
\vspace*{0.7cm}

\institutioneins

\vspace*{0.7cm}
Eine Einsicht in
\ifthenelse{\isundefined{\autorzwei}}{\ifthenelse{\isundefined{\jahr}}{diesen Praxisbeleg}{diese Bachelorthesis}}{diese Projektarbeit}
ist nicht gestattet. Ausgenommen davon sind die betreuenden Dozenten sowie die befugten Mitglieder des Prüfungsausschusses. Eine Veröffentlichung und Vervielfältigung der Arbeit – auch in Auszügen – ist nicht gestattet.

\vspace*{0.7cm}

Ausnahmen von dieser Regelung bedürfen einer schriftlichen Genehmigung des Unternehmens.

\clearpage
\rohead{\textnormal{\headmark}}
}
\vfuzz=0.1pt

\ifthenelse{\isundefined{\jahr}}{}{\chapter*{Themenblatt}
\rohead{\textnormal{Themenblatt}}
\addcontentsline{toc}{chapter}{Themenblatt}
\textcolor{red}{\large{Dies ist ein Platzhalter für das Themenblatt, welches durch den Vorsitzenden des Prüfungsausschusses ausgestellt wird. Diese Seite ersetzen!}}
\normalsize
\normalcolor
\clearpage
\rohead{\textnormal{\headmark}}}

%! Nicht benötigte Verzeichnisse hier auskommentieren
%? Inhaltsverzeichnis
\vfuzz=5pt
\tableofcontents
\newpage
\vfuzz=0.1pt

%? Abbildungsverzeichnis
\listoffigures
\newpage

%? Tabellenverzeichnis
\listoftables
\newpage

%? Programmcodeverzeichnis
\listofcodes
\newpage

%? Formelverzeichnis
\listofformeln
\newpage

%? Abkürzungsverzeichnis
\include{inhalt/Abkürzungen}

%! Hier beginnt der eigentliche Inhalt der Arbeit.
\mainmatter

%! Eine der folgenden Zeilen wieder aktivieren und entsprechende Datei ablegen
%? druckt Firmenlogo ab Kapitel 1 in Kopfzeile - SVG ist, auch bei kleinen Grafiken, zu bevorzugen, wenn vorhanden
%\lohead{\includesvg[height=8mm,inkscapelatex=false]{bilder/firmenlogo}}
%\lohead{\includegraphics[height=8mm]{bilder/firmenlogo}}

% Es ist möglich, die ganze Arbeit in eine Datei (z.B. "Inhalt.tex") zu schreiben,
% allerdings empfiehlt es sich, zur besseren Strukturierung mehrere Dateien, bspw. eine pro Kapitel, zu verwenden.
% Diese werden folgendermaßen eingebunden:
%\include{inhalt/Inhalt}

%! Doku-/Testinhalt, diese Zeile bei Nutzung der Vorlage entfernen
%Alle Kapitel beginnen mit der Hauptüberschrift
\chapter{Einleitung}
Diese PDF wurde mit der Vorlage erstellt, um die Funktion und Formatierung dieser zu zeigen.

Die Vorlage\onlinezitat{Vorlage} ist ein Gemeinschaftsprojekt im Rahmen unseres Studiums.
Der Docker-Container\onlinezitat{HILLE2021} gehört dazu.
Die Vorlage richtet sich weitestgehend nach dem Dokument \ac{HAWA}\onlinezitat{HAWA} der \href{https://www.ba-glauchau.de/}{Staatlichen Studienakademie Glauchau}.

Weitere Hinweise befinden sich in der README.md oder im \href{https://github.com/DSczyrba/Vorlage-Latex/wiki}{Wiki}.
Eine ausführliche Dokumentation zu diesem Dokument wird folgen.

\chapter{Beispiele}
\label{sec:beispiele}
\section{Pixelgrafiken}
Siehe Quelltext.\fn{Dort wird das Kommando \striche{bild} aufgeführt. In dieser Vorlage sind allerdings keine Bilder (im vorgesehenen Ordner) enthalten.}
%\bild[0.5]{dateiname-ohne-endung}{Text zu einem Bild}{label1}
\section{Vektorgrafiken}
Siehe Quelltext.
%\svg[0.5]{dateiname-ohne-endung}{Text zu einer \ac{SVG}-Datei}{label2}
\section{Programmcode}
\begin{code}
    \begin{minted}{python}
        import asdf
        from foo import bar
        
        def yolo():
            return "glhf"
        
        class Wooo(Foo):
            def __init__(self, boo):
                self.doo = boo
                moo = 1 + 2 +3
    \end{minted}
    \caption[Beispielcode]{Beispielcode}
    \label{code:example}
\end{code}

An dieser Stelle wird noch ein einzeiliger Code eingefügt, der keine Zeilennummerierung erhalten soll.
Dies kann genutzt werden, wenn man nur kurz auf eine Funktion eingehen möchte und diese nicht im Quellcodeverzeichnis erscheinen soll.
\mint{python}|print("Hallo Vorlage")|

Ein weiterer Code um zu schauen, ob danach die Zeilennummerierung wieder funktioniert.

\begin{code}
  \begin{minted}{python}
      class Wooo(Foo):
          def __init__(self, boo):
              self.doo = boo
              moo = 1 + 2 +3
  \end{minted}
  \linkcaption{Beispielcode}
  \label{code:example2}
\end{code}
\vgcaption{https://link-wo-es-den-code-gibt.de}{06.07.2022}
\section{Ordnerstruktur}
    In diesem Abschnitt wird eine Ordnerstruktur in \literef{beispielbaum} gezeigt.
    Ordnerstrukturen können im Quellcode definiert werden.
    Die Symbole für Ordner und Dateien können, auf Wunsch, ausgetauscht oder erweitert werden, um verschiedene Dateitypen abzubilden.
    \verzeichnis{%
          .1 \dtfolder Vorlage-Latex. .2 \dtfile HINWEISE.md.
          .2 \dtfile LICENSE.
          .2 \dtfile README.md.
          .2 \dtfile sortieren.py.
          .2 \dtfolder Latex.
            .3 \dtfolder bilder.
              .4 \dtfile firmenlogo.svg.
            .3 \dtfolder inhalt.
              .4 \dtfile Abkürzungen.tex.
              .4 \dtfile Anhang.tex.
            .3 \dtfolder light.
              .4 \dtfile main.tex.
              .4 \dtfile README.md.
              .4 \dtfile vorlage\_light.tex.
            .3 \dtfile literatur.bib.
            .3 \dtfile main.tex.
            .3 \dtfile main\_abstract.tex.
            .3 \dtfile metadaten.sty.
    }{Ein Verzeichnis-Baum}{beispielbaum}

    \section{Formeln}
    Formeln werden mittels \striche{\textbackslash formula} in die Arbeit eingebunden.
    Die Beschriftungen befinden sich, wie es die HAWA\onlinezitat[Abs. 3.3.3.7]{HAWA} verlangt, rechts neben der Formel.
    \formula{$\Delta p_{WZ}=\Delta p\cdot\dfrac{\dot{V}^2_S}{\dot{V}^2_G}$}{%
    $\dot{V}^2_S = $ Spitzendurchfluss $\left[ m^3/h\right]$\\
    $\dot{V}^2_G = $ maximaler Durchfluss im Wasserzähler $\left[ m^3/h\right]$\\
    $\Delta p = $ Druckverlust bei $V_{max} \left[bar\right]$}{Druckverlust}{formel:ohm}

    \section{Überschriften}
    \subsection{in verschiedenen}
    \subsubsection{Größen}
    Kapitelüberschriften werden per \emph{\textbackslash chapter}, Abschnitte per \emph{\textbackslash section}, Unterabschnitte per \emph{\textbackslash subsection} und Unterunterabschnitte per \emph{\textbackslash subsubsection} eingefügt.

\chapter{Test-/Dokutabelle}
Diese Tabelle dient hauptsächlich zum Testen der einzelnen Kommandos, sowie als minimales Beispiel für eine \emph{tabularx}-Tabelle:
\hbadness=10000 %"Tabelle 1" in Spalte Beispiel erzeugt sonst eine Warnung
\begin{table}
\begin{tabularx}{\columnwidth}{|p{3cm}|X|p{.2\columnwidth}|}
\hline
Gegenstand & Beispiel & Befehl \\
\hline
Kurzer Verweis & \literef{sec:beispiele} & \emph{\textbackslash literef}\\
\hline
Langer Verweis & \fullref{beispielbaum} & \emph{\textbackslash fullref}\\
\hline
Verweis ohne Objektname & \autoref{beispieltabelle} & \emph{\textbackslash autoref}\\
\hline
\multicolumn{2}{|c|}{verbundene Spalten mit zentriertem Text} & \emph{\textbackslash multicolumn} \\
\hline
Zeile 1 & \multirow{2}{\hsize}{verbundene Zeilen inklusive automatischer Zeilenumbrüche} & \emph{\textbackslash multirow} \\
\cline{1-1}\cline{3-3}
Zeile 2 & & mit \emph{\textbackslash hsize}\\
\hline
\end{tabularx}
\caption{Beispieltabelle}
\label{beispieltabelle}
\end{table}
\hbadness=1000

In diesem Satz wird eine Quelle mit nur einem Autor zitiert\onlinezitat{HAWA}, eine Quelle mit drei Autoren\vgonlinezitat{Vorlage} wird sinngemäß zitiert und eine Quelle mit vier Autoren\onlinezitat[Abs.~2.3]{rfc3596}, welche als \emph{Autor 1; u.a.} dargestellt werden sollte, existiert ebenfalls. Es folgen ein Buch-Zitat mit Seitenangabe\zitat[9]{KOHM2020} und ein sinngemäßes Zitat aus einer unveröffentlichten Quelle\vguvzitat{UV}.


%? Quellen- und Literaturverzeichnis
% Quellen neigen dazu, zu breite Zeilen durch (z.B.) schlecht umgebrochene URLs zu erzeugen, die sich kaum vermeiden lassen. --> Ignorieren
\hfuzz=10pt
\printbibliography[title=Quellenverzeichnis]
\hfuzz=0.1pt

%? Anhang
%! Anhang

\clearpage
\appendix
\clearpage

%! Chapter Befehl wird umgeschrieben, um Überschriften zu verbergen
%! Kann, falls Überschriften gewollt sind, entfernt oder erst später eingefügt werden.
% Beginn
\makeatletter
\renewcommand{\chapter}[1]{%
\par\refstepcounter{chapter}%
\sectionmark{#1}%
\NR@gettitle{#1}%<---------
\addcontentsline{atoc}{chapter}{\bfseries\protect\numberline{\thechapter}{\mdseries#1}}%
\lohead{\textnormal{#1}}%
}
\makeatother
% Ende

%! Anpassung der Darstellung von Abbildungen im Anhang
%! Eine Variante auskommentieren
%? Möglichkeit 1: ohne Nummerierung
%\renewcommand{\bild}[4][1.0]{\begin{figure}[H]
    %\centering
    %\includegraphics[width=#1\columnwidth]{bilder/#2}
    %\caption*{\bfseries Abbildung \mdseries #3}
    %\label{#4}
    %\end{figure}}

%? Möglichkeit 2: mit Nummerierung aber nicht im Abbildungsverzeichnis
\renewcommand{\bild}[4][1.0]{\begin{figure}[H]
    \centering
    \includegraphics[width=#1\columnwidth]{bilder/#2}
    \caption[]{#3}
    \label{#4}
    \end{figure}}

%! CD/USB-Inhalt, für USB-Stick entsprechend anpassen, Icon ein-/ CD ausblenden
\chapter{Inhalt der CD}
\label{cd-inhalt}
\renewcommand\DTstyle{\sffamily}
\renewcommand{\DTstyle}{\textrm\expandafter\raisebox{-0.7ex}}
\dirtree{%
.1 
%! CD
\begin{tikzpicture}
    \draw circle (0.16);
    \draw circle (0.07);
    \draw circle (0.02);
\end{tikzpicture}
%! USB
%\hspace{.2em}\dtusb
\raisebox{.05em}{ CD mit folgenden Inhalten:}.
.2 \dtfolder Anhänge.
.2 \dtfolder LaTeX-Quellen.
.2 \dtfolder Online-Quellen.
.2 \dtfile dieses Dokument.
.2 \dtfile \href{https://www.youtube.com/watch?v=dQw4w9WgXcQ}{YouTube-Video} als Bonus.
}
\vspace*{\fill}
\begin{center}
    \begin{tikzpicture}
        \draw (0,0) rectangle (12.2,12.2);
        \draw (6.1,6.1) circle (5.5);
        \draw (6.1,6.1) circle (.75);
        \draw (6.1,6.1) circle (2.4);
    \end{tikzpicture}
\end{center}
\vspace*{\fill}
\clearpage

% Warnungen für vorgefertigte Dokumente deaktivieren
\hbadness=10000
% automatische Auswahl der Erklärungen
\ifthenelse{\isundefined{\autorzwei}}{%  Eidesstattliche Erklärung
\cleardoublepage
\lohead{\textnormal{Ehrenwörtliche Erklärung}}
\rohead{}
    \vspace*{1cm}
    \begin{center}
        \huge\textbf{Ehrenwörtliche Erklärung}\\
    \end{center}
    \vspace*{1cm}
    \normalsize
    Ich erkläre hiermit ehrenwörtlich,

    \begin{enumerate}
        \vspace{1cm}
        \item dass ich \ifthenelse{\isundefined{\jahr}}{meinen Praxisbeleg}{meine Bachelorthesis} mit dem Thema:\\

        \textbf{\titel }\\

        ohne fremde Hilfe angefertigt habe,
        \item dass ich die Übernahme wörtlicher Zitate aus der Literatur sowie die\\
        Verwendung der Gedanken anderer Autoren an den entsprechenden\\
        Stellen innerhalb der Arbeit gekennzeichnet habe und
        \item dass ich \ifthenelse{\isundefined{\jahr}}{meinen Praxisbeleg}{meine Bachelorthesis} bei keiner anderen Prüfung vorgelegt habe.\\[1,5cm]
    \end{enumerate}
    Ich bin mir bewusst, dass eine falsche Erklärung rechtliche Folgen haben wird.\\[1,5cm]

    Glauchau, \abgabedatum \newline\noindent\rule{0.35\columnwidth}{0.4pt}\hspace{0.05\columnwidth}\rule{0.6\columnwidth}{0.4pt}\\
    Ort, Datum\hspace{0.27\columnwidth}Unterschrift
}{%
    \ifthenelse{\isundefined{\autordrei}}{\cleardoublepage
\lohead{\textnormal{Ehrenwörtliche Erklärung}}
\rohead{}
    \vspace*{1cm}
    \begin{center}
        \huge\textbf{Ehrenwörtliche Erklärung}\\
    \end{center}
    \vspace*{1cm}
    \normalsize
    Wir erklären hiermit ehrenwörtlich,

    \begin{enumerate}
        \vspace{1cm}
        \item dass wir unsere Projektarbeit mit dem Thema:\\

        \textbf{\titel }\\

        ohne fremde Hilfe angefertigt haben,
        \item dass wir die Übernahme wörtlicher Zitate aus der Literatur sowie die\\
        Verwendung der Gedanken anderer Autoren an den entsprechenden\\
        Stellen innerhalb der Arbeit gekennzeichnet haben und
        \item dass wir unsere Projektarbeit bei keiner anderen Prüfung vorgelegt haben.\\[1,5cm]
    \end{enumerate}
    Wir sind uns bewusst, dass eine falsche Erklärung rechtliche Folgen haben wird.\\[1,5cm]

    Glauchau, \abgabedatum\newline\noindent\rule{0.35\columnwidth}{0.4pt}\hspace{0.05\columnwidth}\rule{0.6\columnwidth}{0.4pt}\\
    Ort, Datum\hspace{0.27\columnwidth}Unterschriften


    \newpage

\begin{table}[H]
    \centering
    \newcolumntype{Y}{>{\centering\arraybackslash}X}
    \begin{tabularx}{\columnwidth}{|X|Y|Y|}
        \hline
        Namen:            & \autoreins  & \autorzwei  \\
        \hline
        Matrikelnummern:  & \matnumeins & \matnumzwei \\
        \hline
        Studiengang:      & \multicolumn{2}{c|}{\studiengang}\\
        \hline
        Titel der Arbeit: & \multicolumn{2}{c|}{\titel}\\
                        %& \multicolumn{2}{c|}{zweite Zeile, falls nötig}\\
        \hline
        Datum:            & \multicolumn{2}{c|}{\abgabedatum}\\
        \hline
        Unterschriften:   &             &\\
                          &             &\\
        \hline
    \end{tabularx}
\end{table}

\vfill
}{%
        \ifthenelse{\isundefined{\autorvier}}{\cleardoublepage
\lohead{\textnormal{Ehrenwörtliche Erklärung}}
\rohead{}
    \vspace*{1cm}
    \begin{center}
        \huge\textbf{Ehrenwörtliche Erklärung}\\
    \end{center}
    \vspace*{1cm}
    \normalsize
    Wir erklären hiermit ehrenwörtlich,

    \begin{enumerate}
        \vspace{1cm}
        \item dass wir unsere Projektarbeit mit dem Thema:\\

        \textbf{\titel }\\

        ohne fremde Hilfe angefertigt haben,
        \item dass wir die Übernahme wörtlicher Zitate aus der Literatur sowie die\\
        Verwendung der Gedanken anderer Autoren an den entsprechenden\\
        Stellen innerhalb der Arbeit gekennzeichnet haben und
        \item dass wir unsere Projektarbeit bei keiner anderen Prüfung vorgelegt haben.\\[1,5cm]
    \end{enumerate}
    Wir sind uns bewusst, dass eine falsche Erklärung rechtliche Folgen haben wird.\\[1,5cm]

    Glauchau, \abgabedatum\newline\noindent\rule{0.35\columnwidth}{0.4pt}\hspace{0.05\columnwidth}\rule{0.6\columnwidth}{0.4pt}\\
    Ort, Datum\hspace{0.27\columnwidth}Unterschriften


    \newpage

\begin{table}[H]
    \centering
    \newcolumntype{Y}{>{\centering\arraybackslash}X}
    \begin{tabularx}{\columnwidth}{|X|Y|Y|Y|}
        \hline
        Namen:            & \autoreins  & \autorzwei  & \autordrei \\
        \hline
        Matrikelnummern:  & \matnumeins & \matnumzwei & \matnumdrei \\
        \hline
        Studiengang:      & \multicolumn{3}{c|}{\studiengang}\\
        \hline
        Titel der Arbeit: & \multicolumn{3}{c|}{\titel}\\
                        %& \multicolumn{3}{c|}{zweite Zeile, falls nötig}\\
        \hline
        Datum:            & \multicolumn{3}{c|}{\abgabedatum}\\
        \hline
        Unterschriften:   &             &             & \\
                          &             &             &\\
        \hline
    \end{tabularx}
\end{table}

\vfill
}{\cleardoublepage
\lohead{\textnormal{Ehrenwörtliche Erklärung}}
\rohead{}
    \vspace*{1cm}
    \begin{center}
        \huge\textbf{Ehrenwörtliche Erklärung}\\
    \end{center}
    \vspace*{1cm}
    \normalsize
    Wir erklären hiermit ehrenwörtlich,

    \begin{enumerate}
        \vspace{1cm}
        \item dass wir unsere Projektarbeit mit dem Thema:\\

        \textbf{\titel }\\

        ohne fremde Hilfe angefertigt haben,
        \item dass wir die Übernahme wörtlicher Zitate aus der Literatur sowie die\\
        Verwendung der Gedanken anderer Autoren an den entsprechenden\\
        Stellen innerhalb der Arbeit gekennzeichnet haben und
        \item dass wir unsere Projektarbeit bei keiner anderen Prüfung vorgelegt haben.\\[1,5cm]
    \end{enumerate}
    Wir sind uns bewusst, dass eine falsche Erklärung rechtliche Folgen haben wird.\\[1,5cm]

    Glauchau, \abgabedatum\newline\noindent\rule{0.35\columnwidth}{0.4pt}\hspace{0.05\columnwidth}\rule{0.6\columnwidth}{0.4pt}\\
    Ort, Datum\hspace{0.27\columnwidth}Unterschriften


    \newpage

\begin{table}[H]
    \centering
    \newcolumntype{Y}{>{\centering\arraybackslash}X}
    \begin{tabularx}{\columnwidth}{|X|Y|Y|Y|Y|}
        \hline
        Namen:            & \autoreins  & \autorzwei  & \autordrei  & \autorvier\\
        \hline
        Mat.-Num.:        & \matnumeins & \matnumzwei & \matnumdrei & \matnumvier\\
        \hline
        Studiengang:      & \multicolumn{4}{c|}{\studiengang}\\
        \hline
        Titel der Arbeit: & \multicolumn{4}{c|}{\titel}\\
                            %& \multicolumn{4}{c|}{zweite Zeile, falls nötig}\\
        \hline
        Datum:            & \multicolumn{4}{c|}{\abgabedatum}\\
        \hline
        Unterschriften:   &              &             &            &\\
                          &              &             &            &\\
        \hline
    \end{tabularx}
\end{table}

\vfill
}
    }
}

% Warnungen zurücksetzen
\hbadness=1000
%!##########################################

\end{document}
