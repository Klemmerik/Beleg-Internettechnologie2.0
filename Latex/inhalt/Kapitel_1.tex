\chapter{Projektplanung}
\label{Projektplanung}

\section{Grundkonzept}
\label{Grundkonzept}

Der Ausgangspunkt für das Projekt ist eine, von der BA-Glauchau bereitgestellte \ac{JSON} Datei.
Die \ac{JSON} enthält die Information, wie viele Modulprüfungen der Studierende bestanden und nicht bestanden hat.
Diese Informationen wird in einer Datenbank gespeichert und dient als Referenzwert.
Wird die \ac{JSON} zu einem späteren Zeitpunkt erneut betrachtet, können die aktuellen Informationen mit den bereits abgespeicherten Informationen der Datenbank verglichen werden.
Dabei können die vier folgenden Fälle auftreten:
\begin{enumerate}
    \item Anzahl der bestandenen und nicht bestandenen Prüfungen ist identisch
    \item Anzahl der bestandenen Prüfungen ist größer
    \item Anzahl der nicht bestandenen Prüfungen ist größer
    \item Anzahl der bestandenen und nicht bestandenen Prüfungen ist größer
\end{enumerate}

Abhängig vom eintreffenden Szenario wird eine entsprechende Benachrichtigung an den Studierenden versendet.
Zusätzlich wird der Datensatz mit den Informationen der aktuellen \ac{JSON} überschrieben und dient als neuer Referenzwert.
Eine Ausnahme ist hierbei der erste Fall.
Bei diesem wird keine Benachrichtigung versendet und der Datensatz nicht aktualisiert.

%Ziel ist es, dieses Grundkonzept in einen Automatisierungsprozess zu implementieren.
%Die Anzahl der Studierenden, welche den Benachrichtigungsservice verwenden, ist nicht bekannt.
%Der Automatisierungsprozess wird so konzipiert, dass eine ansteigende Nutzeranzahl keinen Einfluss auf dessen Funktionsweise besitzt.


\section{Gesamtprozess}
\subsection{Service-Anmeldung}
Damit der Studierende über neu eintreffende Modulnoten informiert werden kann, muss sich dieser für den Benachrichtigungsservice anmelden.
Die Anmeldung für den Benachrichtigungsservice erfolgt über eine Webseite.
Der Studierende wird aufgefordert, die folgenden, für den Benachrichtigungsprozess notwendigen, Daten anzugeben: Name, Matrikelnummer, Hashwert und E-Mail-Adresse.
Nach der erfolgreicher Validierung werden die Daten des Studierenden an eine Datenbank übermittelt und als neuer Datensatz abgespeichert.
Die Anmeldung für den Benachrichtigungsservice ist damit abgeschlossen.

\subsection{Überprüfungsverfahren neuer Modulnoten}
Die Überprüfung auf potentiell eingetroffene Modulnoten, folgt einem festen Schema.
Die Software erhält aus der Datenbank den Datensatz eines Studierenden mit den folgenden Informationen: Name, Matrikelnummer, Hashwert, E-Mail-Adresse sowie die Anzahl bestandener und nicht bestandener Prüfungen.

Mithilfe der Attribute \textit{Matrikelnummer} und \textit{Hashwert} wird eine \ac{URL} für einen HTTP-Reguest zusammengesetzt.
Die Abfrage liefert eine \ac{JSON}-Datei zurück - \fullref{Grundkonzept}.
Es folgt der Vergleich der Anzahl der bestandenen-/ nicht bestandenen Prüfungen aus dem Datensatz mit der Anzahl der bestandenen-/ nicht bestandenen Prüfungen aus der \ac{JSON}-Datei.
Ist die Anzahl der bestandenen-/ nicht bestandenen Prüfungen aus der \ac{JSON}-Datei größer als die des Datensatzes, wird eine entsprechende Benachrichtigung per E-Mail-Adresse versendet.
Die Empfänger E-Mail-Adresse entspricht dabei der E-Mail-Adresse aus dem Datensatz des Studierenden.
Gleichzeitig wird der Datensatz durch die Anzahl der bestandenen-/ nicht bestandenen Prüfungen der \ac{JSON}-Datei aktualisiert.

Im Falle, dass die Attribute \textit{bestandene Prüfungen} und \textit{nicht bestandene Prüfungen} des Datensatzes den Wert NULL aufweisen, handelt es sich um den erstmaligen Überprüfungsversuch nach der Anmeldung eines Studierenden.
In diesem Fall werden die Werte NULL durch die Werte der \ac{JSON}-Datei aktualisiert und eine Bestätigungsemail über die erfolgreiche Anmeldung des Benachrichtigungsservices versendet.

Der gesamte Ablauf wird für jeden Datensatz (Studierenden) der Datenbank genau einmal durchgeführt. 








%was für Ressourcen brauch ich um es so umzusetzen wie es geplant ist?
 
\section{Ablaufdiagramm}