\chapter{Projektplanung}
\label{Projektplanung}

\section{Grundkonzept}
\label{Grundkonzept}

Der Ausgangspunkt für das Projekt ist eine, von der BA-Glauchau bereitgestellte \ac{JSON} Datei.
Die \ac{JSON} enthält die Information, wie viele Modulprüfungen der Studierende bestanden und nicht bestanden hat.
Diese Informationen wird gespeichert und dient als Referenzwert.
Wird die \ac{JSON} zu einem späteren Zeitpunkt erneut betrachtet, können die aktuellen Informationen mit den bereits abgespeicherten Informationen verglichen werden.
Dabei können die vier folgenden Fälle auftreten:
\begin{itemize}
    \item[1.] Anzahl der bestandenen und nicht bestandenen Prüfungen ist identisch
    \item[2.] Anzahl der bestandenen Prüfungen ist größer
    \item[3.] Anzahl der nicht bestandenen Prüfungen ist größer
    \item[4.] Anzahl der bestandenen und nicht bestandenen Prüfungen ist größer
\end{itemize}

Abhängig vom eintreffenden Szenario wird eine entsprechende Benachrichtigung an den Studierenden versendet.
Zusätzlich werden die aktuellen Informationen der \ac{JSON} gespeichert und dienen als neuer Referenzwert.
Eine Ausnahme ist hierbei der erste Fall.
Bei diesem wird keine Benachrichtigung versendet und die Informationen der \ac{JSON} werden nicht überschrieben.

%Ziel ist es, dieses Grundkonzept in einen Automatisierungsprozess zu implementieren.
%Die Anzahl der Studierenden, welche den Benachrichtigungsservice verwenden, ist nicht bekannt.
%Der Automatisierungsprozess wird so konzipiert, dass eine ansteigende Nutzeranzahl keinen Einfluss auf dessen Funktionsweise besitzt.


\section{Konzipierung des Prozessablaufes}
Um die in \literef{Grundkonzept} dargelegte Strategie in einen Automatisierungsprozess zu implementieren, werden definierte Strukturen und Abläufe benötigt.



1. JSON runterladen
2. ist die JSON valide?
3. vergleich daten aus JSON mit Daten der DB
4. jeweiligerr Fall trifft ein
5. nächster User


%was für Ressourcen brauch ich um es so umzusetzen wie es geplant ist?
 
\section{Ablaufdiagramm}