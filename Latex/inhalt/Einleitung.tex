\chapter{Einleitung}
\section{Problemstellung}
Die Ergebnisse einer absolvierten Modulprüfung werden den Studierenden der BA-Glauchau ausschließlich innerhalb der Plattform \textit{Campus-Dual} bereitgestellt.
Nach jeder geschriebenen Prüfung folgt eine Zeitspanne, in der die Prüfung korrigiert, dokumentiert und die Modulnote anschließend veröffentlicht wird.
 
Die Studierenden sind damit gezwungen in regelmäßigen Abständen die Plattform \textit{Campus-Dual} zu besuchen, um zu erfahren, ob eine Modulnote bereits bekannt gegeben wurde.
Eine sich mit jeder Prüfung wiederholende Aufgabe, welche äußerst zeitaufwendig, ineffektiv und mühselig ist.

Die folgende Projektarbeit beschäftigt sich mit der Suche und Umsetzung einer alternativen Lösung, um Studierende zuverlässig und direkt über den Eingang neuer Modulnoten auf der Plattform \textit{Campus-Dual} zu informieren.

\section{Zielstellung}
Den Studierenden der BA-Glauchau wird die Möglichkeit geboten, sich informieren zu lassen, sobald eine neue Modulnote auf der Plattform \textit{Campus-Dual} bereitgestellt wurde.
Der Benachrichtigungsservice ist dabei für jeden Studierenden gleichermaßen verwendbar.
Die Benachrichtigung wird dabei durch einen Automatisierungsprozess realisiert.
Um die Funktionalität des Automatisierungsprozesses zu gewährleisten, wird dieser durch Plausibilitätsprüfungen überwacht und geprüft.
Die persönlichen Daten der Studierenden werden durch Sicherheitsvorkehrungen von externen Zugriffen geschützt.
Um eine nutzerfreundliche Handhabung zu ermöglichen, wird die Einrichtung des Benachrichtigungsservice durch den Studierenden auf ein Minimum reduziert.